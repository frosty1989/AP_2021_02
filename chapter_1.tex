
\section{Classification of robots and their structures}

Industrial robots can be classified according to various criteria: the number of degrees of freedom, kinematic structure, drives used, workspace geometry, motion characteristics, control method or programming method. According to the abovementioned criteria, several types of robots are distinguished:
\subsection{Number of degrees of freedom}

\begin{itemize}
    \item Universal robot - 6 degrees of freedom
    \item Redundant robot - more than 6 degrees of freedom
    \item Deficient robot - less than 6 degrees of freedom
\end{itemize}

\subsection{Kinematic structure}

\begin{itemize}
    \item Serial robots - with an open-loop kinematic chain
    \item Parallel robots - with a closed-loop kinematic chain
    \item Hybrid robots - combining both types of kinematic chains
\end{itemize}


\subsection{Type of drives}
Electric
Hydraulic
Pneumatic

Workspace geometry
Cartesian
Cylindrical
Spherical
Angular
SCARA


Currently, industrial robots with electric drives predominate in numbers. If
high loads are required, hydraulic drives are used and for high speeds pneumatic
drives are preferred.

Before we dive into the programming of industrial robotic arms, let us revise a few concepts.
A workcell represents a robot and a collection of machines or peripherals. A single robot controller is responsible for controlling the various appliances of a workcell.
A robot end effector is a peripheral placed at the end of the robotic arm. The end effector represents the last link of the robot. According to the application, end effectors can be grippers, welding devices, spray guns or grinding and deburring machines.

The robot controller is equipped with a so-called interface software. The interface software makes it easier for the user to program the robotic arm.  
Two basic information need to be programmed into the robotic arm:
position data and 
procedure
Many different ways of teaching a robot position do exist:
Positional commands
Teach pendant
Lead-by-the-nose
Offline programming
Robot simulation tools
Manufacturing independent robot programming tools
