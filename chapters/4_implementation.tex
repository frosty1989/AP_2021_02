\label{chap:implementation}

\epigraph{Python is the "most powerful language you can still read".}{\textit{Paul Dubois}}

\section{FANUC robots programming specifics}

\subsection{FANUC Roboguide}



\subsection{FANUC robots programming languages}

The FANUC company implements two programming languages for programming their robot controllers: Teach Pendant (TP) or KAREL. The Teach Pendant language is mainly used for motion control of the robotic arm and edited via the Pendant. Teach pendant programs are either binary files (.tp file extension) or can be human-readable ASCII files (.ls file extension). The KAREL language is a high-level language and does not support robot movements. KAREL is mainly used to implement algorithms. KAREL programs can be only edited using a Personal Computer, and they can not be edited using a Pendant.

\subsection{Compiling a FANUC TP program}

Only a Teach Pendant program in binary format can be run on FANUC controllers. Because RoboDK creates TP programs as human-readable ASCII files, the Teach Pendant programs need to be converted to binary format before uploading them to the robot controller. Two options to convert .ls programs to .tp programs exist:

\begin{enumerate}
\item The ASCII Upload option must be loaded on the robot controller. After upload an .ls file to the controller it is automatically converted to a .tp file.
\item The program is compiled using the WinOLPC tools from Roboguide.
\end{enumerate}

\section{Importing curves from Solidworks}

To import curves from SolidWorks to RoboDK, use the \mintinline{shell-session}{Settings} control element of the SolidWorks RoboDK plugin and then select your current project using the \mintinline{shell-session}{Load Project...} button. The next step is to click the \mintinline{shell-session}{LoadCurves} button and highlight the curves that define the robot path and the adjacent surfaces of these curves. Lastly, confirm your action by pressing the \mintinline{shell-session}{checkmark}.

\section{RoboDK station}

A RoboDK project containing a robot, robot tools, additional CAD files, robot frames and a robot program is called a station. A RoboDK station is saved as one file (\mintinline{shell-session}{.rdk} extension).  The FANUC M-20iA/35M is used in this project because it is readily available in the RoboDK library and differs from the FANUC M-20iA/20M robotic arm only in payload capacity.

\section{Robot machining projects in RoboDK}

The applications of robot machining in the industry are numerous. Some applications include:

\begin{itemize}

    \item milling
    \item drilling
    \item chamfering
    \item deburring

\end{itemize}

RoboDK offers three types of robot manufacturing projects:

\begin{itemize}

    \item Robot machining project
    \item Curve follow project 
    \item Point follow project 

\end{itemize}

This chapter deals with setting up a Curve follow project in RoboDK. In laser shock peening, the laser (the tool) is static, and the robot holds the object. Therefore, a Curve follow project with a constant tool orientation is set up in RoboDK.

\subsection{Setting up a Curve follow project in RoboDK}


\section{RoboDK API for Python}

\section{Installation, Python setup and path settings of RoboDK API}

\section{Modifying a post processor}


